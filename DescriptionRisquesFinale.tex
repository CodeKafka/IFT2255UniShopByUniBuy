\documentclass[16pt]{report}
\input{preamble}

\title{\Huge{Géni Logiciel}\\{IFT2255}\\{\textbf{Description des cas d'utilisation}}}
\author{\huge{Franz Girardin}}
\date{\today}
\lstset{inputencoding=utf8/latin1}

\usepackage{graphicx}
\usepackage{caption}
\usepackage{subcaption}
\usepackage{arydshln}

\usepackage{balance}
\usepackage{mathpazo}
\usepackage{dirtree}
\usepackage{titlesec}
\titleformat{\chapter}
  {\small\bfseries} % format
  {}                % label
  {0pt}             % sep
  {\huge}           % before-code


\usepackage{lipsum}
\usepackage{titling}
\renewcommand\maketitlehooka{\null\mbox{}\vfill}
\renewcommand\maketitlehookd{\vfill\null}

\newcommand{\varitem}[3][black]{%
  \item[%
   \colorbox{#2}{\textcolor{#1}{\makebox(5.5,7){#3}}}%
  ]
}
\usepackage{afterpage}
\newcommand\myemptypage{
    \null
    \thispagestyle{empty}
    \addtocounter{page}{-1}
    \newpage
    }





% from https://tex.stackexchange.com/a/167024/121799
\newcommand{\ClaudioList}{red,DarkOrange1,Goldenrod1,Green3,blue!50!cyan,DarkOrchid2}
\newcommand{\SebastianoItem}[1]{\foreach \X[count=\Y] in \ClaudioList
{\ifnum\Y=#1\relax
\xdef\SebastianoColor{\X}
\fi
}
\tikz[baseline=(SebastianoItem.base),remember
picture]{%
\node[fill=\SebastianoColor,inner sep=4pt,font=\sffamily,fill opacity=0.5] (SebastianoItem){#1)};}
}
\newcommand{\SebastianoHighlight}{\tikz[overlay,remember picture]{%
\fill[\SebastianoColor,fill opacity=0.5] ([yshift=4pt,xshift=-\pgflinewidth]SebastianoItem.east) -- ++(4pt,-4pt)
-- ++(-4pt,-4pt) -- cycle;
}}    
%====================================================================

%====================================================================
\newcommand*{\authorimg}[1]%
    { \raisebox{-1\baselineskip}{\includegraphics[width=\imagesize]{#1}}}
\newlength\imagesize  

\begin{document}
\fontsize{12}{12}\selectfont
\subsection*{Panne de serveur du site où UniShop est hébergé \textcolor{myb}{\textit{Risque sévère}}}
Lorsque le serveur de la plateforme tombe en panne, UniShop devient inaccessible pour les acheteurs et les revendeurs. 
Cette interruption totale de service entraîne une perte de ventes pour les revendeurs. Des occurrences fréquentes 
pourraient dissuader les revendeurs de continuer à collaborer avec UniShop, les poussant vers la concurrence. 
Pour les acheteurs, cela signifie qu'ils ne peuvent pas acquérir le matériel nécessaire pour leurs cours. 
Ils risquent alors de prendre du retard, ce qui pourrait impacter leurs résultats académiques. Certains pourraient 
se tourner vers des concurrents pour effectuer leurs achats. De plus, les acheteurs ayant une commande active 
ne peuvent pas suivre l'état de leur commande ou la modifier.

\subsection*{Sécurité et protection des données clients \textcolor{myb}{\textit{Risque sévère}}}
Une protection insuffisante des informations personnelles des acheteurs, telles que leurs adresses ou leurs 
informations de paiement, expose UniShop à des risques d'attaques cybernétiques. Les conséquences d'un vol de données 
peuvent être désastreuses, allant du vol d'identité pour les clients à une perte totale de confiance en la plateforme. 
Si les clients perdent confiance en UniShop, les revendeurs pourraient également se détourner de la plateforme.

\subsection*{Gestion erronée de l'inventaire \textcolor{myb}{\textit{Risque important}}}
Un inventaire incorrect sur UniShop peut engendrer divers problèmes. Les revendeurs pourraient ne pas être alertés 
lorsque leurs produits sont épuisés, ce qui perturbe leur planification de réapprovisionnement. De plus, des produits 
indisponibles pourraient être présentés comme disponibles, causant des déceptions chez les acheteurs.

\subsection*{Difficulté à trouver les politiques de retour et d'échange \textcolor{myb}{\textit{Obstacle mineur}}}
Si les politiques de retour et d'échange sont difficiles à trouver, cela pourrait engendrer de la frustration chez 
les acheteurs. Bien que cet obstacle ne bloque pas les principales fonctionnalités d'UniShop, il pourrait diminuer la 
satisfaction des utilisateurs et impacter négativement leur expérience.

\subsection*{Gestion inadéquate d'avis et commentaires d'utilisateurs \textcolor{myb}{\textit{Risque modéré}}}
Une gestion inadéquate des avis et commentaires sur UniShop pourrait éroder la confiance entre acheteurs et revendeurs.
Si le système n'offre pas de modération efficace, de faux avis ou des commentaires malveillants pourraient influencer 
la perception des produits ou du site. Une telle situation risque de pénaliser injustement les revendeurs et de 
décourager la participation active sur la plateforme.

\end{document}


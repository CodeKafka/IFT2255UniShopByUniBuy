\documentclass[16pt]{report}
\input{preamble}

\title{\Huge{Géni Logiciel}\\{IFT2255}\\{\textbf{Description des cas d'utilisation}}}
\author{\huge{Équipe UniBuy} \\\\ \cross \;\; Franz Girardin \;\;\;\; \cross \;  Emmanuel F Gérald Paraison \;\;\;\; \cross \;\;Johann Sourou \;\;\;\;  \cross \;\;Yann Saah}
\date{7 Octobre 2023}
\lstset{inputencoding=utf8/latin1}

\usepackage{graphicx}
\usepackage{caption}
\usepackage{subcaption}
\usepackage{arydshln}

\usepackage{balance}
\usepackage{mathpazo}
\usepackage{dirtree}
\usepackage{titlesec}
\titleformat{\chapter}
  {\small\bfseries} % format
  {}                % label
  {0pt}             % sep
  {\huge}           % before-code


\usepackage{lipsum}
\usepackage{titling}
\renewcommand\maketitlehooka{\null\mbox{}\vfill}
\renewcommand\maketitlehookd{\vfill\null}

\newcommand{\varitem}[3][black]{%
  \item[%
   \colorbox{#2}{\textcolor{#1}{\makebox(5.5,7){#3}}}%
  ]
}
\usepackage{afterpage}
\newcommand\myemptypage{
    \null
    \thispagestyle{empty}
    \addtocounter{page}{-1}
    \newpage
    }



\usepackage{stackengine}
\def\cross{%
  \stackon[1ex]{\rule{0.4pt}{1.5ex}}{\rule{.75ex}{0.4pt}}}
\def\invcross{%
  \stackon[0.5ex]{\rule{0.4pt}{1.5ex}}{\rule{.75ex}{0.4pt}}}
\def\stacktype{L}

% from https://tex.stackexchange.com/a/167024/121799
\newcommand{\ClaudioList}{red,DarkOrange1,Goldenrod1,Green3,blue!50!cyan,DarkOrchid2}
\newcommand{\SebastianoItem}[1]{\foreach \X[count=\Y] in \ClaudioList
{\ifnum\Y=#1\relax
\xdef\SebastianoColor{\X}
\fi
}
\tikz[baseline=(SebastianoItem.base),remember
picture]{%
\node[fill=\SebastianoColor,inner sep=4pt,font=\sffamily,fill opacity=0.5] (SebastianoItem){#1)};}
}
\newcommand{\SebastianoHighlight}{\tikz[overlay,remember picture]{%
\fill[\SebastianoColor,fill opacity=0.5] ([yshift=4pt,xshift=-\pgflinewidth]SebastianoItem.east) -- ++(4pt,-4pt)
-- ++(-4pt,-4pt) -- cycle;
}}    
%====================================================================

%====================================================================
\newcommand*{\authorimg}[1]%
    { \raisebox{-1\baselineskip}{\includegraphics[width=\imagesize]{#1}}}
\newlength\imagesize  

\begin{document}
\maketitle
\pagebreak
\twocolumn





\section*{\textbf{\textcolor{blue}{1- Nom :}} S'inscrire à la plateforme}
\textbf{But :} S'enregistrer comme utilisateur pour avoir accès aux fonctionnalités clé de la plateforme. \\
\textbf{Acteur :} \textit{Utilisateur} (acteur principal) ; \textit{Système de gestion de profil} (acteur secondaire).

\textbf{Scénario principal :}
\begin{enumerate}[leftmargin=4em]
    \item L'utilisateur est dirigé sur la page de création de profil.
    \begin{enumerate}[leftmargin=4em]
        \item Après avoir tenté d'effectuer une commande sans être connecté.
        \item Après avoir cliqué sur le bouton s'inscrire de la page d'accueil.
    \end{enumerate}
    \item L'utilisateur est invité à choisir le type de compte qu'il souhaite créer.
    \item L'utilisateur choisit de crée un compte acheteur.
    \item L'utilisateur entre ses informations personnelles.
    \item Le système de gestion de profil valide les informations personnelles entrées et génère un compte.
    \item Un courriel de confirmation est envoyé à l'adresse de l'utilisateur.
    \item L'utilisateur est renvoyé à la page d'accueil.
\end{enumerate}

\textbf{Scénario alternatif :}
\begin{enumerate}[leftmargin=4em]
    \item[\textcolor{red}{3-a}] L'utilisateur choisit de créer un compte revendeur.
            \begin{enumerate}[leftmargin=4em]
                \item[3-a-1] L'étape 4 du \textit{Scénario principal} est enclenchée.   
            \end{enumerate}
    \item[\textcolor{red}{5-a}] Une entrée d'information personnelle est invalide.
    \begin{enumerate}[leftmargin=4em]
        \item Le nom choisi est déjà présent dans la base de données d'acheteurs.
        \item Le nom choisi est présent dans la base de données revendeurs.
        \item Le courriel choisi est déjà présent dans la base de données d'acheteurs.
        \item Le courriel choisi est déjà présent dans la base de données revendeurs.
        \item L'entrée ne correspond pas au format accepté pour une adresse courriel valide.
        \item Le numéro de téléphone est invalide.
        \item L'adresse du domicile ou de l'entreprise de l'utilisateur est invalide.
        \item L'utilisateur a laissé une case vide.
    \end{enumerate}
    \item[\textcolor{red}{5-b}] L'utilisateur est invité à effectuer une saisie valide.
    \item[\textcolor{red}{6-a}] L'utilisateur reste bloqué sur la page et n'est pas redirigé après avoir complété son inscription.
\end{enumerate}

\section*{\textbf{\textcolor{blue}{2- Nom :}} Se connecter à la plateforme}
\textbf{But :} Permettre à un utilisateur de se connecter à son compte pour effectuer des actions d'acheteur ou de revendeur. \\
\textbf{Acteur :} \textit{Utilisateur} (acteur principal) ; \textit{Système de gestion de profil} (acteur secondaire).

\textbf{Scénario principal :}
\begin{enumerate}[leftmargin=4em]
    \item L'utilisateur est dirigé sur la page de connexion.
    \begin{enumerate}[leftmargin=4em]
        \item Après avoir tenté d'effectuer une commande sans être connecté.
        \item Après avoir cliqué sur le bouton \textit{se connecter}.
    \end{enumerate}
    \item L'utilisateur entre ses informations de connexion.
    \item Le message « Connexion en cours » est affiché durant le traitement des informations.
    \item Le système valide les informations.
    \item L'utilisateur est renvoyé à la page d'accueil en étant connecté à son compte.
\end{enumerate}

\textbf{Scénario alternatif :}
\begin{enumerate}[leftmargin=4em]
    \item[\textcolor{red}{4-a}] La combinaison du nom d'utilisateur et du mot de passe ne correspond à aucun profil de la base de données associée au système de gestion de profil.
    \begin{enumerate}[leftmargin=4em]
        \item Le système ne retrouve pas le compte dans la base de données.
        \item Un message indique au client que les informations entrées sont erronées.
        \item L'utilisateur est renvoyé à l'étape 2.
    \end{enumerate}
    \item[\textcolor{red}{4-b}] Le compte associé aux identifiants est suspendu.
    \begin{enumerate}[leftmargin=4em]
        \item Le système reconnaît le compte mais celui-ci est suspendu.
        \item Un message signale à l'utilisateur que le compte est suspendu.
        \item L'utilisateur est renvoyé à l'écran d'accueil.
    \end{enumerate}
\end{enumerate}


\section*{\textbf{\textcolor{blue}{3- Nom :}} Modifier les informations du compte}

\textbf{But :} Permettre à l'utilisateur de changer ses informations personnelles. \\
\textbf{Acteur :} \textit{Utilisateur} (acteur principal), \textit{système de gestion de profil} (acteur secondaire). \\
\textbf{Précondition :} L'utilisateur est connecté.

\textbf{Scénario principal :}
\begin{enumerate}[leftmargin=4em]
    \item L'utilisateur accède à la section « Mon Profil ».
    \item Il clique sur le bouton « Modifier ».
    \item Des champs éditables avec ses informations actuelles sont affichés.
    \item L'utilisateur change les informations qu'il souhaite modifier.
    \item Il clique sur « Sauvegarder ».
    \item Le système valide les modifications.
    \item Un message de confirmation est affiché.
\end{enumerate}

\textbf{Scénario alternatif :}
\begin{enumerate}[leftmargin=4em]
    \item[\textcolor{red}{6-a}] Le système détecte des informations non valides (par exemple : un format de téléphone incorrect).
    \begin{enumerate}[leftmargin=4em]
        \item[6-a-1] Un message d'erreur est affiché.
        \item[6-a-2] L'utilisateur est invité à corriger ses informations.
    \end{enumerate}
\end{enumerate}

\section*{\textbf{\textcolor{blue}{4- Nom :}} Ajouter un produit au panier}

\textbf{But :} Permettre à l'utilisateur d'ajouter des produits à son panier. \\
\textbf{Acteur :} \textit{Acheteur} (utilisateur principal), \textit{Système d'inventaire} (utilisateur secondaire). \\
\textbf{Précondition :} L'utilisateur a trouvé un produit qu'il souhaite acheter.

\textbf{Scénario principal :}
\begin{enumerate}[leftmargin=4em]
    \item L'utilisateur sélectionne le produit souhaité.
    \item Il choisit la quantité.
    \item Il clique sur « Ajouter au panier ».
    \item Le produit est ajouté à son panier.
    \item Un message de confirmation s'affiche.
\end{enumerate}

\textbf{Scénario alternatif :}
\begin{enumerate}[leftmargin=4em]
    \item[\textcolor{red}{4-a}] La quantité demandée n'est pas disponible.
    \begin{enumerate}[leftmargin=4em]
        \item[4-a-1] Un message d'erreur s'affiche.
        \item[4-a-2] L'utilisateur est invité à réduire la quantité ou à choisir un autre produit.
    \end{enumerate}
\end{enumerate}

\section*{\textbf{\textcolor{blue}{5- Nom :}} Retirer un produit du panier}

\textbf{But :} Permet à l’utilisateur de retirer des produits dans son panier pour ajuster sa commande. \\
\textbf{Acteur :} \textit{Acheteur} (utilisateur principal), \textit{Système d'inventaire} (utilisateur secondaire). \\
\textbf{Précondition :} L'utilisateur possède au moins un produit dans son panier et se trouve sur la page répertoriant 
les produits du panier.

\textbf{Scénario principal :}
\begin{enumerate}[leftmargin=4em]
    \item L’acheteur clique sur le bouton Retirer du panier du produit à éliminer.
    \item Un message d’alerte demande à l’utilisateur de confirmer l’action.
    \item L’utilisateur confirme le retrait du produit.
    \item Le produit est retiré du panier de l’utilisateur.
    \item Un message prévient l’utilisateur du succès du retrait du produit de son panier
    \item Le système d’inventaire met à jour le nombre d’instances du produit disponibles.
\end{enumerate}

\textbf{Scénario alternatif :}
\begin{enumerate}[leftmargin=4em]
    \item[\textcolor{red}{3-a}] L’utilisateur abandonne le processus de retrait de produit.
    \begin{enumerate}[leftmargin=4em]
        \item[3-a-1] Un message informe le client que le produit est toujours disponible dans son panier.
        \item[3-a-2] L’utilisateur est redirigé vers la page répertoriant les produits du panier
    \end{enumerate}
\end{enumerate}




\section*{\textbf{\textcolor{blue}{6- Nom :}} Passer une commande}

\textbf{But :} Permet à l'utilisateur d'acheter les produits dans son panier et d'enclencher le processus de livraison. \\
\textbf{Acteur :} \textit{Acheteur} (acteur principal), \textit{Système de création de commande} (acteur secondaire), \textit{Système de payement} (secondaire). \\
\textbf{Précondition :} Le panier de l'acheteur contient au minimum de 1 article et l'utilisateur se trouve sur la page répertoriant les produits de son panier.

\textbf{Scénario principal :}
\begin{enumerate}[leftmargin=4em]
    \item L’acheteur clique sur le bouton réaliser une commande.
    \item Le système de gestion de commande vérifie si l’utilisateur est connecté.
    \item Le client sélectionne les produits de son panier qu’il désire acheter.
    \item Une page s’affiche et demande les informations nécessaires pour réaliser une livraison.
    \item Le système de gestion de commande propose à l’utilisateur d’utiliser les informations associées à son compte
    \item L’utilisateur accepte d’utiliser les informations associées à son compte.
    \item Le système de gestion de commande vérifie les informations de livraison
    \item Les informations de livraison sont valides.
    \item Le client est redirigé vers la page de paiement.
    \item Le client entre les informations nécessaire pour réaliser le payement.
    \item Le système de payement vérifie les informations entrées par l’acheteur.
    \item Le système de paiement valide les informations de paiement. 
    \item Une requête visant à débiter le compte est envoyé.
    \item Un message informe le client que la commande est validée.
    \item Un numéro de commande est généré par le système de création de commande.
    \item Les détails de la commande sont affichés.
    \item Un courriel résumant la transaction est envoyé à l’acheteur.
    \item Le système de gestion de commande met à jour l'état de la commande à en production 
    \item Le système signale le revendeur.
    \end{enumerate}

\textbf{Scénario alternatif :}
\begin{enumerate}[leftmargin=4em]
    \item[\textcolor{red}{2-a}] L’utilisateur n’est pas connecté.
            \begin{enumerate}
        \item[2-a-1.] Le système de gestion de commande enclenche les étapes de connexion (cas d’utilisation 2).
             \end{enumerate}
         \item[\textcolor{red}{2-b}] L’utilisateur est connecté.
            \begin{enumerate}
            \item[2-b-1] Le système de gestion de commande enclenche l’étape 3 du scénario principal.
            \end{enumerate}
        \item[\textcolor{red}{6-a}] L’utilisateur refuse d’utiliser les informations associées à son compte
            \begin{enumerate}
            \item[6-a-1] Le système de gestion de commande demande à l’utilisateur de saisir les 	informations nécessaires pour passer la commande 
            \item[6-a-2] Le système de création de commande enclenche l’étape 7 du scénario principal. 
            \end{enumerate}

        \item[\textcolor{red}{8-a}] Une des informations entrées est invalide.
            \begin{enumerate}
            \item[8-a-1] Un message d’alerte indique laquelle des informations est incorrecte.
            \item[8-a-2] L’utilisateur est invité à saisir de nouvelles informations valides. 
             \end{enumerate} 

         \item[\textcolor{red}{10-b}] L’utilisateur annule la transaction.
            \begin{enumerate}
            \item[10-b-1] L’Utilisateur est dirigé vers la page d’affichage de son panier d’articles
            \end{enumerate}
        \item[\textcolor{red}{12-a}] Les informations de paiement sont invalides.
            \begin{enumerate}    
            \item[12-a-1] Un message informe le client.
            \item[12-a-2] La page bloque et redemande à l’utilisateur de rentrer de nouvelles informations.
            \end{enumerate}
        \item[\textcolor{red}{14-a}] Montant présent sur la carte virtuelle insuffisant.
            \begin{enumerate}
            \item[14-a-1] Un message indique que le payement est refusé.
            \item[14-a-2] L’acheteur est renvoyé à la page de saisie d’information bancaire.
            \end{enumerate}

    % Et ainsi de suite pour les étapes restantes...
\end{enumerate}


\section*{\textbf{\textcolor{blue}{7- Nom :}} Annuler une commande}

\textbf{But :} Permet à l'utilisateur d'annuler une commande initiée. \\
\textbf{Acteur :} \textit{Utilisateur} (acteur principal), \textit{Revendeur} (acteur secondaire), \textit{système de paiement} (acteur secondaire). \\
\textbf{Précondition :} Avoir déjà effectué une commande. Se trouver dans le menu répertoriant l'ensemble des commandes complétées ou en cours.

\textbf{Scénario principal :}
\begin{enumerate}[leftmargin=4em]
    \item L’utilisateur sélectionne une commande à annuler à partir du menu répertoriant les commandes associées à son profil.
    \item L’acheteur clique sur le bouton Annuler la commande.
    \item L’acheteur confirme son choix.
\end{enumerate}

\textbf{Scénario alternatif :}
\begin{enumerate}[leftmargin=4em]
    \item[\textcolor{red}{3-a}] La commande est à un stade trop avancé pour être annulée (acheminement en cours).
\end{enumerate}


\section*{\textbf{\textcolor{blue}{8- Nom :}} Changer l’état d’une commande}

\textbf{But :} Garder le client informé de l’évolution de la commande  
\textbf{Acteur :} \textit{revendeur} (acteur principal), \textit{système de gestion d’inventaire} (acteur secondaire), \textit{acheteur} (acteur secondaire), \textit{compagnie d’expédition} (acteur secondaire). \\
\textbf{But :} garder le client informé de l'évolution de la commande. \\
\textbf{Précondition :} Le revendeur se connecte à la plateforme ; la commande dont le revendeur veut changer l’état a déjà été initiée par l’acheteur.

\textbf{Scénario principal :}
\begin{enumerate}[leftmargin=4em]
    \item Le revendeur est redirigé vers la page affichant son profil après avoir cliqué sur un bouton destiné à cette action.
    \item Le revendeur est redirigé vers la page affichant ses commandes après avoir cliqué sur le bouton consulter les commandes.
    \item Le revendeur sélectionne la commande à consulter.
    \item Le revendeur sélectionne l’option modifier l’état d’une commande.
    \item Le revendeur change le statut :  en production → en livraison (transmet un numéro de suivit à l’expéditeur).
    \item Le système de gestion d’inventaire met à jour la quantité d’instances disponibles pour cet article.
    \item Le revendeur transmet les informations de l’expéditeurs à l’acheteur.
    \item Un courriel est envoyé à l’acheteur pour l’informé.
\end{enumerate}

\textbf{Scénario alternatif :}
\begin{enumerate}[leftmargin=4em]
    \item[\textcolor{red}{5-a}] Le revendeur change le statut : commande en livraison→ livré après avoir reçu la confirmation de l’acheteur.
\end{enumerate}


\section*{\textbf{\textcolor{blue}{9- Nom :}} Créer un produit}

\textbf{But :} Permet à un revendeur d’enregistrer sur la plateforme un produit qu’il souhaite vendre. \\
\textbf{Acteur :} Revendeur (principal), Système d’inventaire (secondaire). \\
\textbf{Précondition :} L’utilisateur est connecté à son compte UniShop en tant que revendeur. L’utilisateur se trouve dans la page de son profil.

\textbf{Scénario principal :}
\begin{enumerate}[leftmargin=4em]
    \item Le revendeur clique sur le bouton Vendre un produit.
    \item Le revendeur rentre les informations du produit qu’il souhaite vendre.
    \item Le revendeur fournit la catégorie du produit 
    \item Le revendeur fournit une description du produit
    \item Le revendeur fournit le nom du produit
    \item Le revendeur fournit la marque du produit
    \item Le revendeur fournit le modèle du produit
    \item Le revendeur fournir la quantité initiale du produit
    \item Le revendeur est invité à ajouter une image pour le produit
    \item Le revendeur accepte d’ajouter une image.
    \item L’image inclut est validée (format adéquat).
    \item Le revendeur est invité à ajouter une vidéo pour le produit.  
    \item Le revendeur accepte d’ajouter une vidéo. 
    \item La vidéo ajoutée est validée (format adéquat).
    \item Le système d’inventaire enregistre le produit et crée les instances. 
    \item Le revendeur est renvoyé vers la page de son profil.

\end{enumerate}

\textbf{Scénario alternatif :}
\begin{enumerate}[leftmargin=4em]
    \item[\textcolor{red}{2-a}] Le système le rejette si les informations sont invalides ou incomplètes.
            \begin{enumerate}
                    \item[2-a-1.] Un message indique au revendeur que les informations sont invalides.
            \end{enumerate} 

        \item[\textcolor{red}{10-a}] Le revendeur refuse d’ajouter une image
            \begin{enumerate}
                    \item[10-a-1] L’étape 12 du scénario principal est enclenché
            \end{enumerate}


        \item[\textcolor{red}{12-a}] Le revendeur refuse d’ajouter une vidéo
            \begin{enumerate}
                \item[10-a-1] L’étape 15 du scénario principal est enclenché
            \end{enumerate}
        
        \item[\textcolor{red}{11-a}] L’image inclut est invalide 
            \begin{enumerate}
                \item[10-a-1] Le revendeur est invité à fournir un format adéquat ou à sauter l’étape d’ajout d’image.
            \end{enumerate}

        \item[\textcolor{red}{14-a}] La vidéo inclut est invalide 
            \begin{enumerate}
                \item[14-a-1] Le revendeur est invité à fournir un format adéquat ou à sauter l’étape d’ajout de vidéo.
            \end{enumerate}
\end{enumerate}



\section*{\textbf{\textcolor{blue}{10- Nom:}} Proposer un échange}

\textbf{But :} Permet à un revendeur de proposer un échange de produit à un acheteur comme moyen de régler un problème. \\
... % et ainsi de suite, vous pouvez remplir les sections similaires à la précédente
\textbf{Acteur :} Revendeur(principal), Acheteur(secondaire), Système d’inventaire(secondaire), système de création de commande (acteur secondaire)\\  
\textbf{Précondition :} L’utilisateur est connecté à son compte UniShop en tant que revendeur. L’acheteur a signalé un problème au revendeur.

\textbf{Scénario principal :}
\begin{enumerate}[leftmargin=4em]
    \item Le revendeur est redirigé vers la page affichant son profil après avoir cliquée sur un bouton destiné à cette action
    \item Le revendeur est redirigé vers la page affichant ses commandes après avoir cliqué sur le bouton consulter les commandes
    \item Le revendeur clique sur le bouton « Proposer un échange ».
    \item Le revendeur choisi le produit à renvoyer dans l’historique de commande de l’acheteur.
    \item Le système d’inventaire vérifie si le produit est disponible.
    \item Le système de création de commande envoie un message à l’acheteur l’avisant de l’offre d’échange.
\end{enumerate}

\textbf{Scénario alternatif}
\begin{enumerate}
    \item[\textcolor{red}{3-a}] Le produit sélectionné n’est plus disponible (rupture de stock). 
        \begin{enumerate}
            \item[3-a-1] Un message informe le revendeur que le produit n’est pas disponible.
            \item[3-a-1] Le revendeur est renvoyé à l’étape 2.
        \end{enumerate}
\end{enumerate}



\section*{\textbf{\textcolor{blue}{11- Cas d'utilisation :}} Aimer un produit}
\textbf{But :} Permet à un acheteur de donner une note à une produit et de laisser un commentaire.\\
... % et ainsi de suite, vous pouvez remplir les sections similaires à la précédente
\textbf{Acteur :} Acheteur(principal), Système de métrique(secondaire). \\ 
\textbf{Précondition :} L’utilisateur est connecté à son compte sur la plateforme. L’utilisateur se trouve sur la page de catalogue de produit. 

\textbf{Scénario principal :}
\begin{enumerate}[leftmargin=4em]
        \item L’utilisateur clique sur le bouton « évaluer le produit » après avoir sélectionné un produit précis.
        \item Le système de métrique acheteur vérifie si l’acheteur a déjà acheté le produit.
        \item La page d’évaluation du produit s’affiche.
        \item L’utilisateur est invité à laisser une évaluation pour le produit
        \item L’utilisateur accepte de laisser une évaluation
        \item Une boîte de texte apparaît pour que l’utilisateur rédige son évaluation 
        \item L’utilisateur confirme l’envoi de son évaluation 
        \item L’utilisateur est invité à laisser un commentaire pour le produit
        \item L’utilisateur accepte de laisser un commentaire
        \item Une boîte de texte apparaît pour que l’utilisateur rédige un commentaire
        \item L’utilisateur confirme l’envoi 
        \item Le système de métrique met à jour les métriques de l’acheteur et du revendeur ciblé par l’action de l’utilisateur. 
        \item L’utilisateur est redirigé vers le menu de catalogue de produit

\end{enumerate}

\textbf{Scénario alternatif}
\begin{enumerate}
    \item[\textcolor{red}{2-a}] L'utilisateur n’as jamais acheter le produit.  
        \begin{enumerate}
            \item[2-a-1] Un message informe le client qu’il ne peut pas évaluer le produit.
        \end{enumerate}
        
        \item[\textcolor{red}{7-a}] L’utilisateur annule le processus d’évaluation.
        \begin{enumerate}
            \item[7-a-1] L’étape 12 en enclenchée.
        \end{enumerate} 

        \item[\textcolor{red}{9-a}] L’utilisateur refuse de laisser une évaluation.
        \begin{enumerate}
            \item[9-a-1] L’étape 12 est enclenchée.
        \end{enumerate}

        \item[\textcolor{red}{11-a}] L’utilisateur annule le processus d’évaluation.
        \begin{enumerate}
            \item[9-a-1] L’étape 12 est enclenchée
        \end{enumerate}




\end{enumerate}


\end{document}

\end{document}


